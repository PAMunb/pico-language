\documentclass{beamer}

\usepackage{listings}

\title{The PICO Programming Language}
\author{Rodrigo Bonif\'{a}cio}

% Define Language
\lstdefinelanguage{pico} {
                    % list of keywords
                    morekeywords={
                      begin, end,
                      if,
                      while, do, od,
                      natural, string, declare, then, else
                    },
                    sensitive=false, % keywords are not case-sensitive
                    morecomment=[l]{//}, % l is for line comment
                    morecomment=[s]{/*}{*/}, % s is for start and end delimiter
                    morestring=[b]'' % defines that strings are enclosed in double quotes
                    }


\begin{document}

\begin{frame}
\titlepage
\end{frame}

\begin{frame}

  \texttt{PICO} is a small programming language
  formally specified at CWI/Amsterdam~\cite{pico-spec}. 
  \pause
  \vskip+2em

  \begin{quote}
    ``It is so simple that its specification fits on a few pages.''
  \end{quote}
  \flushright {\scriptsize{Paul Klint's Course on Advanced Programming}}
  
\end{frame}


\begin{frame}
  \frametitle{Types and Expressions}

  \begin{block}{Types}
    \begin{itemize}
    \item natural numbers
    \item strings
    \end{itemize}
  \end{block}

  \begin{block}{Expressions}
    \begin{itemize}
    \item (+) and (-) for natural numbers
    \item ($\|$) for combining two strings  
    \end{itemize}
  \end{block}
\end{frame}

\begin{frame}
  \frametitle{Statements}

  \begin{itemize}
   \item \textsc{if-then} 
   \item \textsc{if-then-else}
   \item \textsc{while-do}  
  \end{itemize}
\end{frame}

\begin{frame}[fragile]
\begin{block}{A PICO program}  
\begin{scriptsize}
  \begin{lstlisting}[language=pico]
begin
 declare input: natural, output: natural,
     repnr: natural, rep : natural;

 input := 14;
 output := 1;

 while(input - 1) do
  rep := output;
  repnr := input;
  while(repnr - 1) do
   output := output + rep;
   repnr := repnr - 1;
  od
  input := input - 1;
 od
end
\end{lstlisting}
\end{scriptsize}
\end{block}
\end{frame}

\begin{frame}
  \frametitle{What are the elements of a programming language?}

  \pause
  \begin{block}{Each programming language has}
  \begin{itemize}
    \item a (concrete) syntax
    \item a (well-defined) semantics
    \item and a set of practices for its use (pragmatics)   
  \end{itemize}
  \end{block}
  \pause

  However, we might say that, to implement a (small) language,
  all we have to do is to provide an {\color{blue}interpreter} to
  that language. \pause An interpreter specifies an operational
  semantics of a language. 
\end{frame}

\begin{frame}

  Therefore, providing an interpreter for the \textsc{PICO} language
  is our first task. To this end, we have to {\color{blue}design} a
  suitable representation of a \textsc{PICO} program and a set
  of functions to evaluate a \textsc{PICO} program.
  
\end{frame}


\begin{frame}
  Our second challenge is to implement some other language
  processing tools for \textsc{PICO}, such as:

  \begin{itemize}
   \item parser
   \item pretty-printer
   \item a control flow graph generator  
   \item static analysis for computing metrics
   \item static analysis for finding bugs  
  \end{itemize}
\end{frame}

\begin{frame}[allowframebreaks]
  \frametitle{References}
  \bibliographystyle{alpha}
  \bibliography{references.bib}
\end{frame}

\begin{frame}
\titlepage
\end{frame}



\end{document}
